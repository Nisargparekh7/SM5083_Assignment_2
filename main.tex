\documentclass[journal,12pt,twocolumn]{IEEEtran}
\usepackage{tikz}
\usepackage{amsmath}
\usepackage{breqn}
\usepackage{amssymb}
\pagestyle{empty}
\usepackage{setspace}
\usepackage{gensymb}
\singlespacing
\usepackage{mathtools} 
\usepackage{amsmath}
\usepackage{amsthm}
\begin{document}
\providecommand{\sbrak}[1]{\ensuremath{{}\left[#1\right]}}
\providecommand{\lsbrak}[1]{\ensuremath{{}\left[#1\right.}}
\providecommand{\rsbrak}[1]{\ensuremath{{}\left.#1\right]}}
\providecommand{\brak}[1]{\ensuremath{\left(#1\right)}}
\providecommand{\lbrak}[1]{\ensuremath{\left(#1\right.}}
\providecommand{\rbrak}[1]{\ensuremath{\left.#1\right)}}
\providecommand{\cbrak}[1]{\ensuremath{\left\{#1\right\}}}
\providecommand{\lcbrak}[1]{\ensuremath{\left\{#1\right.}}
\providecommand{\rcbrak}[1]{\ensuremath{\left.#1\right\}}}
\newcommand{\myvec}[1]{\ensuremath{\begin{pmatrix}#1\end{pmatrix}}}
\newcommand{\cmyvec}[1]{\ensuremath{\begin{pmatrix*}[c]#1\end{pmatrix*}}}
\newcommand{\mydet}[1]{\ensuremath{\begin{vmatrix}#1\end{vmatrix}}}
\newcommand{\proj}[2]{\textbf{proj}_{\vec{#1}}\vec{#2}}
\let\StandardTheFigure\thefigure
\let\vec\mathbf

\title{
Assignment - 1
}
\author{ Nisarg Parekh \\SM21MTECH14002}
\maketitle
\newpage
\bigskip
\bibliographystyle{IEEEtran}
\section*{\textbf{Problem}}
\noindent
\textbf{\textsl{1.Show that the area of triangle formed by lines
$$x\cos\alpha + y\sin\alpha =p,x\cos\beta + y\sin\beta =q,$$$$x\cos\gamma + y\sin\gamma =r$$
 $$is~~\frac{\left[ p\sin(\beta-\gamma) +q\sin(\gamma-\alpha)+r sin(\alpha-\beta)\right]^2 }{2\sin(\beta-\gamma) \sin(\gamma-\alpha)\sin(\alpha-\beta)}$$ 
 }}
\noindent
\section*{\textbf{Solution}}
\noindent
\begin{align*}
 Area ~formed~ by~ lines:   x\cos\alpha + y\sin\alpha -p=0, \\~x\cos\beta + y\sin\beta -q=0~x\cos\gamma + y\sin\gamma -r=0 is 
\end{align*}
\begin{multline*}
     a_{1}x + b_{1}y +c_{1}=0 \\
    a_{2}x + b_{2}y +c_{2}=0 \\
    a_{3}x + b_{3}y +c_{3}=0 \\
\end{multline*}
\begin{multline}
     =\frac{det \begin{bmatrix}a_{1}&b_{1}&c_{1}\\a_{2}&b_{2}&c_{2}\\a_{3}&b_{3}&c_{3}\end{bmatrix}^{2}}{2~C1~C2~C3}\\\
where ~C1,C2,C3~ are~ co factor ~of~ c1,c2,c3 ~\\\ respectively~ in~ the ~above~ matrix.
\end{multline}

\begin{multline}
     =\frac{det \begin{bmatrix}\cos\alpha&\sin\alpha&-p\\\cos\beta&\sin\beta&-q\\\cos\gamma&\sin\gamma&-r\end{bmatrix}^{2}}{2~C1~C2~C3}\\\
\end{multline}
let's  find Determinate first
\begin{multline}
     Determinate=-\begin{vmatrix} \cos\alpha&\sin\alpha&p\\\cos\beta&\sin\beta&q\\\cos\gamma&\sin\gamma&r \end{vmatrix}\\\because~taking~ common ~ C3 \xleftrightarrow{{-}} C3
\end{multline}
\begin{multline}
     = -[ ~
     \cos\alpha(r\sin\beta-q\sin\gamma) -\\ \sin\alpha(r\cos\beta - q\cos\gamma) + p(\cos\beta \sin\gamma-\sin\beta\cos\gamma)
    ~ ]\\ \because ~expanding~ determinant
\end{multline}

\begin{multline}
     = -[ ~
     (r\cos\alpha\sin\beta-q\sin\gamma\cos\alpha) -\\ (r\sin\alpha\cos\beta - q\cos\gammasin\alpha) + p(\cos\beta \sin\gamma-\sin\beta\cos\gamma)
    ~ ]
\end{multline}
\begin{multline}
     = -[ ~
     r(\cos\alpha\sin\beta- \sin\alpha\cos\beta)-\\ q(\sin\gamma\cos\alpha+ \cos\gammasin\alpha) + p(\cos\beta \sin\gamma-\sin\beta\cos\gamma)
    ~ ]\\ \because ~rearranging ~terms
\end{multline}
\begin{multline}
     = -[ ~
     -r\sin(\alpha-\beta)- q\sin(\gamma-\alpha) - p \sin(\beta-\gamma)
    ~ ]\\ 
\because \sin A\cos B - \sin B\cos A=\sin(A-B)
\end{multline}
\begin{multline}
   Determinate  = [ ~
     r\sin(\alpha-\beta)+ q\sin(\gamma-\alpha) + \\p \sin(\beta-\gamma)
    ~ ]
\end{multline}

Now we will find C1,C2,C3
\begin{multline}
    C1=\begin{vmatrix}\cos\beta & \sin\beta\\ \cos\gamma&\sin\gamma \end{vmatrix}=\cos\beta\sin\gamma -\cos\gamma\sin\beta\\ =-\sin(\beta-\gamma)
\end{multline}
\begin{multline}
    C2=\begin{vmatrix}\cos\alpha & \sin\alpha\\ \cos\gamma&\sin\gamma \end{vmatrix}=\cos\alpha\sin\gamma -\cos\gamma\sin\alpha\\ =\sin(\gamma-\alpha)
\end{multline}
\begin{multline}    
C3=\begin{vmatrix}\cos\alpha & \sin\alpha\\ \cos\beta&\sin\beta \end{vmatrix}=\cos\alpha\sin\beta -\cos\beta\sin\alpha\\ =-\sin(\alpha-\beta)
\end{multline}

Putting equation number 8,9,10,11 in equation 2
\begin{multline}
= \frac{ [r\sin(\alpha-\beta)+ q\sin(\gamma-\alpha) + p \sin(\beta-\gamma)]^2}{2~(-\sin(\beta-\gamma))~(\sin(\gamma-\alpha))~(-\sin(\alpha-\beta))}    
\end{multline}
\begin{multline}
    \frac{\left[ p\sin(\beta-\gamma) +q\sin(\gamma-\alpha)+r sin(\alpha-\beta)\right] }{2\sin(\beta-\gamma) \sin(\gamma-\alpha)\sin(\alpha-\beta)}\\=R.H.S
\end{multline}
putting~ value~ of~p=-1,~q=-1 ,r=-1$~\alpha=60~\beta=45 \\ \gamma=30$

\begin{multline}
=\frac{(-0.17638)^2}{-0.06698}=-0.004644198\\
\end{multline}
\end{document}